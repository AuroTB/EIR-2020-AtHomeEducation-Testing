\documentclass[10pt,spanish,aspectratio=1610]{beamer}
\usepackage[utf8]{inputenc}
\usepackage{amsmath}
\usepackage{graphicx}
\usepackage{amssymb}
\usepackage[spanish]{babel}
\spanishdecimal{.}
\usepackage{subfig}
\usepackage{fancyhdr}
\usepackage{pstricks}
\usepackage{verbatim}
\usepackage[ruled]{algorithm2e}
%\usepackage{ragged2e}
%\usepackage[natbibapa]{apacite}
% \bibliographystyle{apacite} % This is the style you should use with `apacite`.
%\justifying
\DeclareMathOperator{\atantwo}{atan2}
\newcommand\ddfrac[2]{\frac{\displaystyle #1}{\displaystyle #2}}
\usetheme{Boadilla}
\setbeamercovered{transparent}
\beamertemplatenavigationsymbolsempty
\setbeamertemplate{frametitle}
{
  \leavevmode
  \hbox{
  \begin{beamercolorbox}[wd=0.6\paperwidth,left]{frametitle}
    \usebeamerfont{frametitle}\insertframetitle
  \end{beamercolorbox}
  \begin{beamercolorbox}[wd=0.4\paperwidth,center]{frametitle}
    \usebeamerfont{frametitle}\hfill\small{\thesection. \insertsection}
  \end{beamercolorbox}
  }
}
\setbeamertemplate{footline}
{
  \leavevmode%
  \hbox{%
    \begin{beamercolorbox}[colsep=-0.5pt,wd=.33\paperwidth,ht=3ex,dp=1.5ex,center]{author in head/foot}%
      \usebeamerfont{author in head/foot}\insertshortauthor~~ (\insertshortinstitute)
    \end{beamercolorbox}%
    \begin{beamercolorbox}[colsep=-0.5pt,wd=.34\paperwidth,ht=3ex,dp=1.5ex,center]{date in head/foot}%
      \usebeamerfont{author in head/foot}\insertshorttitle
    \end{beamercolorbox}%
    \begin{beamercolorbox}[colsep=-0.5pt,wd=.33\paperwidth,ht=3ex,dp=1.5ex,right]{author in head/foot}%
      \usebeamerfont{author in head/foot}\insertshortdate{}\hspace*{2em}\scriptsize{\insertframenumber{}}\hspace*{1ex}
    \end{beamercolorbox}
  }
}

\begin{document}
\renewcommand{\tablename}{Tabla}
\renewcommand{\figurename}{Figura}

\title[Robocup@Home Beginners]{Curso Introductorio para la Categoría\\Robocup@Home Beginners}
\author[Marco Negrete y Luis González]{Instructores: \\ Marco Antonio Negrete Villanueva \\ Luis González Nava}
\institute[FI, UNAM]{Facultad de Ingeniería, UNAM}
\date[EIR 2020]{Escuela de Invierno de Robótica 2020, Saltillo, México.}

\begin{frame}
\titlepage
\end{frame}

\begin{frame}\frametitle{La categoria @Home}
  
\end{frame}

\begin{frame}\frametitle{La categoria @Home Beginners}
  
\end{frame}

\begin{frame}\frametitle{Hardware necesario}
  \begin{itemize}
  \item Base móvil (se recomienda omnidireccional)
    \begin{itemize}
    \item Peoplebot
    \item Festo
    \item Construída
    \end{itemize}
  \item Cámara (de preferencia, RGB-D)
    \begin{itemize}
    \item Kinect
    \item Intel Real Sense
    \end{itemize}
  \item Sensor Láser (se puede sustituir con cámara RGB-D)
    \begin{itemize}
    \item Hokuyo
    \item Lidar
    \item Generalmente se opera con el nodo \texttt{urg\_node}y debe publicar el tópico \texttt{/scan} de tipo \texttt{sensor\_msgs/LaserScan}
    \item Si no se dispone de uno, las lecturas se pueden simular con el paquete \texttt{pointcloud\_to\_laserscan} pero no es muy recomendable.
    \end{itemize}
  \item Manipulador
    \begin{itemize}
    \item Motores Dynamixel
    \item Kuka
    \end{itemize}
  \item Laptop
  \end{itemize}
\end{frame}

\begin{frame}\frametitle{Configuraciones mínimas}
  \begin{itemize}
    \item Se requiere de un marco de referencia absoluto, comúnmente llamado \texttt{map}. En Rviz, \texttt{map} se selecciona como referencia global.
  \item Se requiere de un archivo que describa la cinemática del robot (archivo \texttt{urdf}), es decir, el árbol de transformaciones. Se recomienda que el \textit{frame} raíz tenga el nombre \texttt{base\_link}.
    \begin{itemize}
    \item Tiene formato \texttt{xml} y se puede hacer a mano o con herramientas como \texttt{xacro}
    \item \url{http://wiki.ros.org/urdf/Tutorials}
    \item \url{http://wiki.ros.org/xacro}
    \item Ejemplo en este repositorio: \texttt{catkin\_ws/src/hardware/robot\_description/robotino.urdf}
    \end{itemize}
  \item La base móvil debe publicar su odometría y aceptar comandos de movimiento.
    \begin{itemize}
    \item Para la odometría, debe publicar la transformación de \texttt{odom} a \texttt{base\_link}.
    \item Para los comandos de movimiento, debe suscribirse al tópico \texttt{/cmd\_vel} de tipo \texttt{geometry\_msgs/Twist}.
    \end{itemize}
  \item Se requiere de un nodo que publique la transformación de \texttt{odom} a \texttt{map}.
    \begin{itemize}
    \item Si se está construyendo un mapa, esta transformación la publican paquetes como \texttt{gmapping} o \texttt{hector-mapping}.
    \item Si ya se tiene un mapa, la trasnformación la publica el nodo de localización, generalmente \texttt{amcl}. 
    \end{itemize}
  \item El archivo de descripción \textit{urdf} debe contener las transformaciones (o una secuencia de ellas) de \texttt{base\_link} a los marcos de referencia del láser, cámara y demás sensores.
  \item El archivo \texttt{catkin\_ws/src/bring\_up/launch/robotino\_simul.launch} es un ejemplo donde se lanza todo lo anterior. 
  \end{itemize}
\end{frame}

\begin{frame}\frametitle{Ejercicio 1}
  Ejecutar el comando \texttt{roslaunch bring\_up robotino\_simul.launch}.
  
  Detener la ejecución y modificar el archivo \texttt{catkin\_ws/src/bring\_up/launch/robotino\_simul.launch} para cambiar lo siguiente:
  \begin{itemize}
  \item Cambiar la descripción del robot (\texttt{robotino.urdf} o \texttt{justina\_simple.urdf})
  \item Cambiar el mapa del ambiente (Universum, Biorobotica o TMR\_2019)
  \end{itemize}
  Modificar el archivo \texttt{catkin\_ws/src/hardware/robot\_description/robotino.urdf} y ver qué sucede cuando:
  \begin{itemize}
  \item Se cambian los valores de la etiqueta \texttt{origin} en la línea 114. 
  \end{itemize}
\end{frame}

\begin{frame}\frametitle{Navegación}
  Dar algo de teoría. 
\end{frame}

\begin{frame}\frametitle{Navegación}
  Ejercicios moviéndole a los params
\end{frame}

\begin{frame}\frametitle{Navegación}
  Recomendaciones para hacer una navegación propia. 
\end{frame}

\begin{frame}\frametitle{Reconocimiento de voz con Pocketsphinx}
  \begin{itemize}
  \item Hay varios repositorios para ROS:
    \begin{itemize}
    \item \url{https://github.com/Pankaj-Baranwal/pocketsphinx}
    \end{itemize}
  \item El usuario debe estar agregado al grupo \textit{audio}: \texttt{sudo usermod -a -G audio <user\_name>}
  \item Documentación en \texttt{https://cmusphinx.github.io/}
  \item Se puede hacer reconocimiento usando una lista de palabras, un modelo de lenguaje o una gramática.
  \item Se utilizarán gramáticas y sus correspondientes diccionarios.
  \item Para construir diccionarios, visitar \texttt{https://cmusphinx.github.io/wiki/tutorialdict/}
  \item Para construir gramáticas, visitar \texttt{https://www.w3.org/TR/2000/NOTE-jsgf-20000605/}
  \end{itemize}
\end{frame}

\begin{frame}\frametitle{Ejercicio}
  Hacer una nueva gramática para una prueba simple de GPSR.
  
\end{frame}

\begin{frame}\frametitle{Síntesis de voz con SoundPlay}
  
\end{frame}

\begin{frame}\frametitle{OpenCV: Segmentación por color}
  Dar algo de teoría
\end{frame}

\begin{frame}\frametitle{OpenCV: Segmentación por color}
  Cosas de ROS y ejercicios
\end{frame}

\begin{frame}\frametitle{OpenCV: Reconocimiento con SIFT}
  Dar algo de teoría
\end{frame}

\begin{frame}\frametitle{OpenCV: Reconocimiento con SIFT}
  Cosas de ROS y ejercicios
\end{frame}

\begin{frame}\frametitle{Planeación de acciones}
Dar opciones: FSM, clips, MDP.  
\end{frame}

\begin{frame}\frametitle{Planeación de acciones. Ejercicio}
Hacer programa para obedecer un comando de navegar a un lugar y buscar a una persona.
\end{frame}
\end{document}